目前常见的补偿空气折射率的方法主要分为两类:直接法和间接法。光的某些成像特性是与波长紧密关联的,利用这些特性就可以从成像结果直接反推出空气折射率,这种方法称为直接法,例如荷兰爱因霍芬科技大学使用的抽气法等。而间接法又可以称为PTF法\cite{高精度空气折射率测量系统设计与实现},是利用对应的传感器间接测量出环境的温度、气压、湿度等影响空气折射率的因素,然后使用经验公式计算出空气折射率,常见的方法有Edlen公式、Birch公式等,Edlen公式补偿方法已广泛用于激光干涉仪的环境误差补偿。
