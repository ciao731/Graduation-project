
\documentclass[type=master]{fduthesis}
% 模板选项:
%   type = doctor|master|bachelor  论文类型,默认为本科论文
%   oneside|twoside                论文的单双面模式,默认为 twoside
%   draft = true|false             是否开启草稿模式,默认关闭
% 带选项的用法示例:
%   \documentclass[oneside]{fduthesis}
%   \documentclass[twoside, draft=true]{fduthesis}
%   \documentclass[type=bachelor, twoside, draft=true]{fduthesis}

\fdusetup{
  % 参数设置
  % 允许采用两种方式设置选项:
  %   1. style/... = ...
  %   2. style = { ... = ... }
  % 注意事项:
  %   1. 不要出现空行
  %   2. “=” 两侧的空格会被忽略
  %   3. “/” 两侧的空格不会被忽略
  %   4. 请使用英文逗号 “,” 分隔选项
  %
  % style 类用于设置论文格式
  style = {
    % font = times,
    % 西文字体(包括数学字体)
    % 允许选项:
    %   font = garamond|libertinus|lm|palatino|times|times*|none
    %
    % cjk-font = fandol,
    % 中文字体
    % 允许选项:
    %   cjk-font = adobe|fandol|founder|mac|sinotype|sourcehan|windows|none
    %
    % 注意:
    %   1. 中文字体设置高度依赖于系统。各系统建议方案:
    %        windows:cjk-font = windows
    %        mac:    cjk-font = mac
    %        linux:  cjk-font = fandol(默认值)
    %   2. 除 fandol 和 sourcehan 外,其余字体均为商用字体,请注意版权问题
    %   3. 但 fandol 字体缺字比较严重,而 sourcehan 没有配备楷体和仿宋体
    %   4. 这里中西文字体设置均注释掉了,即使用默认设置:
    %        font     = times
    %        cjk-font = fandol
    %   5. 使用 font = none / cjk-font = none 关闭默认字体设置,需手动进行配置
    %
    font-size = 5,
    % 字号
    % 允许选项:
    %   font-size = -4|5
    %
    % fullwidth-stop = catcode,
    % 是否把全角实心句点 “.” 作为默认的句号形状
    % 允许选项:
    %   fullwidth-stop = catcode|mapping|false
    % 说明:
    %   catcode   显式的 “。” 会被替换为 “.”(e.g. 不包括用宏定义保存的 “。”)
    %   mapping   所有的 “。” 会被替换为 “.”(使用 LuaLaTeX 编译则无效)
    %   false     不进行替换
    %
    footnote-style = xits,
    % 脚注编号样式
    % 允许选项:
    %   footnote-style = plain|libertinus|libertinus*|libertinus-sans|
    %                    pifont|pifont*|pifont-sans|pifont-sans*|
    %                    xits|xits-sans|xits-sans*
    % 默认与西文字体保持一致
    %
    % hyperlink = color,
    % 超链接样式
    % 允许选项:
    %   hyperlink = border|color|none
    %
    % hyperlink-color = default,
    % 超链接颜色
    % 允许选项:
    %   hyperlink-color = default|classic|material|graylevel|prl
    %
    bib-backend = bibtex,
    % 参考文献支持方式
    % 允许选项:
    %   bib-backend = bibtex|biblatex
    %
    % bib-style = numerical,
    % 参考文献样式
    % 允许选项:
    %   bib-style = author-year|numerical|<其他样式>
    % 说明:
    %   author-year  著者—出版年制
    %   numerical    顺序编码制
    %   <其他样式>   使用其他 .bst(bibtex)或 .bbx(biblatex)格式文件
    %
    % cite-style = {},
    % 引用样式
    % 默认为空,即与参考文献样式保持一致
    % 仅适用于 biblatex;如要填写,需保证相应的 .cbx 格式文件能被调用
    %
    bib-resource = {main_fudan.bib},
    % 参考文献数据源
    % 可以是单个文件,也可以是用英文逗号 “,” 隔开的一组文件
    % 如果使用 biblatex,则必须明确给出 .bib 后缀名
    %
    % logo = {fudan-name.pdf},
    % 封面中的校名图片
    % 模版已自带,通常不需要额外配置
    %
    % logo-size = {0.5\textwidth},      % 只设置宽度
    % logo-size = {{}, 3cm},            % 只设置高度
    % logo-size = {8cm, 3cm},           % 设置宽度和高度
    % 设置校名图片的大小
    % 通常不需要调整
    %
    % auto-make-cover = true
    % 是否自动生成论文封面(封一)、指导小组成员名单(封二)和声明页(封三)
    % 除非特殊需要(e.g. 不要封面),否则不建议设为 false
  },
  %
  % info 类用于录入论文信息
  info = {
    title = {基于粒子群算法的双频激光干涉仪环境误差的软硬件补偿方法},
    % 中文标题
    % 长标题建议使用 “\\” 命令手动换行(不是指在源文件里输入回车符,当然
    % 源文件里适当的换行可以有助于代码清晰):
    %   title = {最高人民法院、最高人民检察院关于适用\\
    %            犯罪嫌疑人、被告人逃匿、死亡案件违法所得\\
    %            没收程序若干问题的规定},
    %
    title* = {Software and hardware compensation method for environmental error of dual frequency laser interferometer based on particle swarm optimization algorithm},
    % 英文标题
    %
    author = {廖永超},
    % 作者姓名
    %
    % author* = {Your name},
    % 作者姓名(英文 / 拼音)
    % 目前不需要填写
    %
    supervisor = {张志平\quad 青年研究员},
    % 导师
    % 姓名与职称之间可以用 \quad 打印一个空格
    %
    major = {微电子学与固体电子学},
    % 专业
    %
    degree = academic,
    % 学位类型
    % 允许选项:
    %   degree = academic|professional
    % 说明:
    %   academic      学术学位
    %   professional  专业学位
    %
    department = {工程与应用技术研究院},
    % 院系
    %
    student-id = {20210860011},
    % 作者学号
    %
    % date = {2021 年 1 月 1 日},
    % 日期
    % 注释掉表示使用编译日期
    %
    % secret-level = ii,
    % 密级
    % 允许选项:
    %   secret-level = none|i|ii|iii
    % 说明:
    %   none  不显示密级与保密年限
    %   i     秘密
    %   ii    机密
    %   iii   绝密
    %
    % secret-year = {五年},
    % 保密年限
    % secret-level = none 时该选项无效
    %
    instructors = {
      {张\quad 三 \quad 教\quad 授},
      {李\quad 四 \quad 教\quad 授},
      {王五六     \quad 研究员}
    },
    % 指导小组成员
    % 使用英文逗号 “,” 分隔
    % 如有需要,可以用 \quad 手工对齐
    %
    keywords = {不确定关系, 量子力学, 理论物理},
    % 中文关键字
    % 使用英文逗号 “,” 分隔
    %
    keywords* = {Uncertainty principle, quantum mechanics, theoretical physics},
    % 英文关键字
    % 使用英文逗号 “,” 分隔
    %
    clc = {O413.1},
    % 中图分类号
    %
    % jel = {C02},
    % JEL 分类号,仅适用于经济学院等部分院系
  }
}

% 需要的宏包可以自行调用

\usepackage{hyperref}
\usepackage{physics}
\usepackage{caption}
\usepackage{subfigure} %插入多图时用子图显示的宏包
\usepackage{graphicx} %插入图片的宏包
\usepackage{url}
\usepackage{float}%设置图片浮动位置的宏包
\usepackage{multirow}
\usepackage{amsmath}
\usepackage{diagbox}
% \usepackage{graphix}
\usepackage[thmmarks]{ntheorem}
% 需要的命令可以自行定义
\newcommand{\hilbertH}{\symcal{H}}
\newcommand{\ee}{\symrm{e}}
\newcommand{\ii}{\symrm{i}}
% Maximum fraction of area at the bottom to place floats, default 0.3.
\renewcommand{\bottomfraction}{0.99}
% Maximum fraction of area at the top to place floats, default 0.7.
\renewcommand{\topfraction}{0.99}
% Maximum fraction of area at the top to place floats spanning two columns.
\renewcommand{\dbltopfraction}{0.99}
% Maximum fraction of area to place floats, default 0.5.
\renewcommand{\floatpagefraction}{0.99}
% Maximum fraction of area to place floats spanning two columns.
\renewcommand{\dblfloatpagefraction}{0.99}
% Minimum fraction of area to place texts, default 0.2.
\renewcommand{\textfraction}{0.45}

% Maximum number of floats at bottom of one page.
\setcounter{bottomnumber}{99}
% Maximum number of floats at top of one page.
\setcounter{topnumber}{99}
% Maximum number of floats at one page.
\setcounter{totalnumber}{99}
% Maximum number of floats spanning two columns at one page.
\setcounter{dbltopnumber}{99}

