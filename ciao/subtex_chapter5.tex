\chapter{粒子群算法的硬件加速设计}
\section{硬件设计方法}
\subsection{流水线技术与握手控制}
当前一般用吞吐率(throughput)作为硬件的性能指标,对于将要设计的粒子群加速模块,其吞吐率的计算公式可以使用下式:
\begin{equation}\label{eq:吞吐率计算}
    throughput = \frac{Operation}{second}=\frac{PSO}{instructions}\times \frac{instructions}{clock cycle} \times \frac{clock cycle}{time(1s)}
    \end{equation}

式\eqref{eq:吞吐率计算}的含义为每秒钟能执行的操作数量,它由三部分组成:有效操作(PSO计算)占所有指令的比例$\frac{PSO}{instructions}$、每个周期能执行的指令数量$\frac{instructions}{clock cycle}$以及1秒钟内包含的周期个数$\frac{instructions}{clock cycle}$,其中前两项为强相关,可以合并为一个因素进行考虑,即一个周期能执行的PSO操作数量,所以最终决定设计出来的硬件的吞吐率的影响指标为:每个周期能执行的PSO操作数量和1秒钟内包含的周期个数。如果不采用流水线技术设计一个单周期的加速器,那么每一个周期能执行的PSO操作数量为1,但是由于一次PSO操作需要涉及到适应度计算、位置和速度更新、种群信息更新三个步骤,而这中间由涉及到很多乘法,这会导致设计出来的加速器的信号延迟很大,从而使得一个周期所需要的时间增加,一秒钟内包含的周期个数较少,最后设计出来的加速器的吞吐率较低,所以需要采用流水线技术。

流水线技术的本质就是通过插入寄存器,将一个较长的组合逻辑分割成多个较短的组合逻辑,并在每个组合逻辑中使用寄存器暂存数据,信号在一个周期内必须通过的路径长度减小了,系统的时钟周期就可以降低了。这就好像是在流水线上的工人,原本一件事情一个工人需要10秒钟才能完成,现在将这件事情分给10个工人去干,每个工人只需要1秒钟即可完成,这样就减少了所需要的时间,所以称为流水线技术。并且这并不会导致每个周期能执行的操作数量下降,为采用流水线技术时,一个周期能完成的PSO操作数量为1,现在采用5级流水线技术,假设所需要进行的PSO操作数量为N,那么一个周期能完成的PSO操作数量为$\frac{N}{N+5}$,根据极限的原理可知,当N足够大时,计算出来的结果仍为1。所以采用流水线技术,根据式\eqref{eq:吞吐率计算}可知,加速器的吞吐率即可以得到提升。