\chapter{全文总结与展望}
\section{本文工作总结}
%\section{论文创新点总结}
本文以光刻机工件台为研究背景,以永磁同步直线电机为例,针对精密直线运动平台的定位力、摩擦力、电磁非线性以及模型参数摄动等问题,研究了滑模控制在精密直线运动平台控制系统中的应用潜力,分别从系统动力学建模与参数辨识、基于递推最小二乘的滑模控制以及基于神经网络的滑模控制方法等方面展开了研究,研究工作中主要取得了以下成果:


1) 建立了精密直线运动平台详细的动力学模型,分析了其扰动类型和经典的定位力与摩擦力的模型,并进一步地根据扰动的特性将扰动分为时变的和时不变的两部分,作为后续控制方法设计的基础。同时用白噪声辨识的方法对实验所用精密直线运动平台进行了系统辨识,得到了系统模型参数,这为控制方法设计提供了一定的依据。

3) 提出了一种基于改进型RLS的积分滑模控制方法。该方法利用RLS能够实时在线估计参数的优势,对精密直线运动平台系统逆模型进行实时估计,并引入定位力模型对定位力主要成分进行自适应补偿,显著提高了精密直线运动平台控制系统的控制性能,实现了更高的位置跟踪精度和扰动抑制能力。

4) 提出了一种多核神经网络动态边界层滑模控制方法。该方法充分考虑了精密直线运动平台的系统特性,尤其是其定位力与摩擦力的扰动形式,并将其考虑到神经网络核函数的设计中,除了高斯核函数,引入了三角核函数和sigmoid核函数分别拟合和补偿定位力和摩擦力带来的影响,显著地提高了系统的扰动抑制能力。提出的动态边界层滑模控制避免了固定边界层滑模控制不能使系统状态轨迹渐近收敛到滑模流形的问题,有效地提高了精密直线运动平台的位置跟踪精度。

5) 搭建了基于精密直线运动平台的控制系统原理验证硬件平台,对所提出的IRLSISMC方法和MNNSMC方法进行了实验验证与分析,结果表明所提方法与同类型的传统TRLSISMC方法和ASMC方法相比,能够极大地提高精密直线运动平台的位置跟踪精度和扰动抑制能力。其中IRLSISMC方法在不同速度的轨迹输入情况下,其RMSE能够维持在4\,$\upmu$m附近,MAE能够维持在11\,$\upmu$m,MA、MSD也能够保持较好的水平,200\,$\upmu$m三阶轨迹跟踪实验中,MA能够维持在6\,$\upmu$m附近,MSD能够维持在5\,$\upmu$m附近;MNNSMC方法在正弦参考轨迹输入与三阶S轨迹输入情况下都能有效地提高系统的跟踪性和鲁棒性,尤其是高速三阶轨迹跟踪情况下,与ASMC相比,MNNSMC方法在匀速段的位置跟踪精度显著提高,其MA可达$2.23\,\upmu$m、MSD可达$3.05\,\upmu$m,充分说明了所提方法的有效性。
\section{研究工作展望}
本文是以光刻机工件台的高性能运动控制为研究背景,以其核心零部件精密直线运动平台为研究对象,以解决实际工程应用难题为研究目标,所提出的控制方法在实验平台上取得了良好的效果,但是由于实验条件有限,未能够在实际工程应用装备中进行实验测试,因此,为了能够更有效地解决实际工程应用问题,后续可以在实际工程应用装备中对所提方法进行验证测试。此外,本文探索了基于递推最小二乘的积分滑模控制方法与基于神经网络的滑模控制方法在精密直线运动平台中的应用前景,提出的改进型RLS方法和MNNSMC方法仍然可以结合其他的控制理论进一步完善,比如可以充分考虑系统的特性,加入一些特定的模型到递推最小二乘的回归向量中,进一步提高系统的位置跟踪性能和扰动抑制能力。多核神经网络可以与其他控制方法相结合,充分发挥多核神经网络同时考虑无模型特性和有模型特性的优势。另外,动态边界层的设计也可以进一步优化,保证系统状态轨迹渐近收敛到滑模流形的同时,提高系统的抗干扰能力。
